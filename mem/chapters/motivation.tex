\section{Description of the Master Thesis}
In this Master Thesis, the main intention is to develop and evolutionary algorithm for solving the well-known \textit{Menu Planning Problem~(MPP)}. The MPP is an optimisation problem which is based on design menu plan under some restrictions. Although there are a lot of different kind of algorithms for solving such a problem, a high percentage of published papers use evolutionary algorithm approaches or any other type of meta-heuristics due its large benefits like robustness, reliability, global-search ability and its simplicity\cite{SELJAK2009414, Moreira2018, Kahraman:2005:HDM:1102256.1102345, Kashima2009, 7257195}. 

Concretely, this Master Thesis will be focused on solving the MPP formulation proposed in \cite{Miranda2018} with an evolutionary algorithm called \textit{Multi-objective Evolutionary Algorithm based on Decomposition} and compare its performance with other state-of-art algorithms such as \textit{Nondominated Sorting Genetic Algorithm II~(NSGA-II)} and \textit{Strength Pareto Evolutionary Algorithm 2~(SPEA 2)}.

\section{State of the art}

The Menu Planning Problem \textit{(MPP)} is a well-known NP-Hard which has been trying to computerise since 1960\cite{Ngo2016}. In essence, the MPP is to find a set of dishes combination which satisfies some restrictions of budge, variety and nutritional requirements for a \textit{n} days sequence. In addition, it can include other constraints such as user preferences, cooking time or the number for meals each day.
Even though there is not consensus about the number of objectives that a MPP's formulation may have, in almost every formulation the cost of the menu plan is considered as one of the main objectives to optimise\cite{Ngo2016, Moreira2018} but it also supports other objective functions like maximising the variability or minimising the cooking time.

Furthermore, the MPP can be studied as a multi-objective problem\cite{Seljak2009} if the amounts of nutrients requirements and cost of the meals are considered independent objectives. This approach leads to reduce the MPP to a Multi-dimensional Knapsack Problem \textit{(MDKP)} where the maximum amount of each nutrient define the limit of the multiple dimensions. However, the MPP is also studying as a single-objective problem where mainly research define the objective function as the total cost of the meals. For instance, a single-objective approach for the MPP is in\cite{Moreira2018} where the authors proposed an evolutionary approach to solving the 5-day Single-Objective Menu Planning Problem composed by three meals daily, using as a function to minimise the total cost of the designed menus. In addition, the set of constraints that the researchers defined to this problem are moderately different from the usual constraints set for the typical MPP. In this occasion, the authors set the student age group, the school category, school duration time, school location, variety of preparations, the maximum amount to be paid for each meal and finally, the lower and upper limits of macro-nutrients as the constraints set to be satisfied for each solution to be considered feasible. Within this research, the authors used the generic Genetic Algorithm (GA) for the computational experiments. The results obtained from the generic GA where compared with a Greedy-based approach. The results prove that the GA outperforms the Greedy-based approach when the limit values of the meals are fixed at R\$ 2.00 for breakfast, R\$ 4.00 for lunch and R\$ 2.00 for the snack. (BRL - R\$ 1.0 ~~ USD - \$ 0.31).

In the other hand, in the paper\cite{Funabiki2011}, the authors refer to the Two-phase Cooking N-day Menu Planning Problem where the objective is to maximise the preferences among the selected foods in the menu plan. The conditions which shape the set of constraints that must be satisfied are only three. The total cooking time of any day must not exceed the limit specified, only foods that which allow two-phase cooking can be selected for two-phase cooking and finally, the food cannot be repeated more times than a certain repeat constraint. In order to face this problem, the researchers used a simple greedy method prioritising the user-specified preference with the cooking time of each food.

Eventually, another study where the MPP is faced as a single-objective problem is\cite{Sufahani2014}. Here, the authors set up a mathematical model to solve the MPP considering only one objective function. The model's goal is to minimise the budget provided by the government subject to the restriction of trying to maximise the variety of dishes. Furthermore, the model tries to create menus in such a way they maximise the nutritional requirements. For the computational experiments, the researchers programmed an Integer Programming algorithm in Matlab using LPSolve. Furthermore, the given results, taking into account that the optimal solution was found within one second, are better compared to other heuristics like GA.

As can be seen, there is a certain variety within the optimisation methods for solving the single-objective MPP approach. Despite that, Evolutionary Computing \textit{(EC)} techniques, such as GA, are mostly cited in the related bibliography as a good choice\cite{Ngo2016, Seljak2009, Moreira2018}.

Therefore, this Master Thesis is focused on develop, analyse and compare an evolutionary algorithm for solving the Menu Planning Problem with other evolutionary algorithms 
focusing on a real world case.