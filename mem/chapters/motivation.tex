\section{Description of the Master Thesis}
In this Master Thesis, the main intention is to develop an evolutionary algorithm for solving the well-known \textit{Menu Planning Problem~(MPP)}. The MPP is an optimisation problem which is based on designing menu plans under some restrictions. Although there is a lot of different kinds of algorithms for solving such a problem, a high percentage of published papers use evolutionary computation due its large benefits like robustness, reliability, global-search ability and its simplicity\cite{SELJAK2009414, Moreira2018, Kahraman:2005:HDM:1102256.1102345, Kashima2009, 7257195}. 

Concretely, this Master Thesis will be focused on solving a recently proposed MPP formulation with an evolutionary algorithm called \textit{Multi-objective Evolutionary Algorithm based on Decomposition}\cite{Zhang2007} and compare its performance with other state-of-art algorithms such as \textit{Nondominated Sorting Genetic Algorithm II~(NSGA-II)}\cite{996017} and \textit{Strength Pareto Evolutionary Algorithm 2~(SPEA 2)}\cite{Laumanns2001SPEA2}.

\section{State of the art}

The Menu Planning Problem \textit{(MPP)} is a well-known NP-Hard problem, which was firstly proposed in 1960\cite{Ngo2016}. In essence, the MPP consists of finding a set of dishes combination which satisfies some restrictions of budge, variety and nutritional requirements for a period of\textit{n} days. In addition, it can include other constraints such as user preferences, cooking time or the number of meals each day.
Even though there is not consensus about the number of objectives that a MPP's formulation may have, in almost every formulation the cost of the menu plan is considered as one of the main objectives to be optimised\cite{Ngo2016, Moreira2018}. But, it also supports other objective functions, like maximising the variability or minimising the cooking time.

Furthermore, the MPP can be studied as a multi-objective problem\cite{Seljak2009} if the amounts of nutrient requirements and cost of the meals are considered as independent objectives. This approach leads to reduce the MPP to a Multi-dimensional Knapsack Problem \textit{(MDKP)} where the maximum amount of each nutrient define the limits of the multiple dimensions. However, the MPP has also been studied as a single-objective problem where the total cost of the meals is considered as the typical objective function. For instance, a single-objective approach for the MPP is in\cite{Moreira2018}. In this particular research, the authors proposed an evolutionary approach to solving the 5-day Single-Objective Menu Planning Problem composed by three meals daily. In addition, the set of constraints that the researchers defined to this problem are moderately different from the usual constraints set for the typical MPP. In this occasion, the authors set the student age group, the school category, school duration time, school location, variety of preparations, the maximum amount to be paid for each meal and finally, and the lower and upper limits of macro-nutrients as the constraints set to be satisfied for each solution to be considered feasible. Within this research, the authors used the standard Genetic Algorithm (GA) for the computational experiments. The results obtained compared with a Greedy-based approach demonstrated that the GA was able to outperform the Greedy-based approach when the limit values of the meals are fixed at R\$ 2.00 for breakfast, R\$ 4.00 for lunch and R\$ 2.00 for the snack. (BRL - R\$ 1.0 ~~ USD - \$ 0.31).

At the same time, in \cite{Funabiki2011}, the authors referred to the Two-phase Cooking N-day Menu Planning Problem where the objective is to maximise the preferences among the selected foods in the menu plan. The conditions which shape the set of constraints that must be satisfied are three. The total cooking time of any day must not exceed the limit specified, only foods that which allow two-phase cooking can be selected for two-phase cooking and finally, the food cannot be repeated more times than a certain repetition constraint. In order to face this problem, the researchers used a simple greedy method prioritising the user-specified preferences with the cooking time of each food.

Eventually, another study where the MPP is faced as a single-objective problem was considered in\cite{Sufahani2014}. Here, the authors set up a mathematical model to solve the MPP considering only one objective function. The goal of the model is to minimise the budget provided by the government subject to the restriction of trying to maximise the variety of dishes. Furthermore, the model tries to create menus in such a way they maximise the nutritional requirements. For the computational experiments, the researchers implemented an Integer Programming algorithm in Matlab using LPSolve. Furthermore, the given results, taking into account that the optimal solution was found within one second, are compared to other heuristics, like GA.

As it can be seen, there is a certain variety within the optimisation methods for solving the single-objective MPP approach. Despite that, Evolutionary Computation \textit{(EC)} techniques, such as GA, are mostly cited in the related bibliography as a suitable choice\cite{Ngo2016, Seljak2009, Moreira2018}.

This Master Thesis is focused on developing, analysing and comparing different well-known multi-objective evolutionary algorithms for solving real-world instances of a multi-objective variant of the MPP.